\documentclass{article}
\usepackage[spanish]{babel}
\usepackage{hyperref}

\title{Guía de Instalación y Configuración de EasyEffects para Filtros de Audio en Ubuntu}
\author{Usuario}
\date{\today}

\begin{document}

\maketitle

\section{Introducción}
Esta documentación describe el proceso de instalación y configuración de EasyEffects para aplicar filtros de audio al micrófono en Ubuntu, específicamente para reducción de ruido y mejora de voz.

\section{Requisitos del Sistema}
\begin{itemize}
    \item Sistema Operativo: Ubuntu (versiones 20.04 o superiores)
    \item Terminal de comandos
    \item Permisos de administrador (sudo)
\end{itemize}

\section{Proceso de Instalación}

\subsection{Paso 1: Actualizar lista de paquetes}
\begin{verbatim}
sudo apt update
\end{verbatim}

\subsection{Paso 2: Instalar EasyEffects}
\begin{verbatim}
sudo apt install easyeffects
\end{verbatim}

\subsection{Paso 3: Instalar complementos LSP (recomendado)}
\begin{verbatim}
sudo apt install lsp-plugins
\end{verbatim}

\section{Configuración}

\subsection{Paso 4: Iniciar EasyEffects}
\begin{verbatim}
easyeffects
\end{verbatim}

\subsection{Paso 5: Configurar dispositivo de entrada}
\begin{itemize}
    \item Abrir la pestaña \textbf{Input}
    \item Seleccionar el micrófono en \textbf{Dispositivo de Entrada}
\end{itemize}

\subsection{Paso 6: Aplicar filtros de audio}
\begin{itemize}
    \item Clic en \textbf{Add Effects}
    \item Seleccionar \texttt{Noise Reduction}
    \item Seleccionar \texttt{Speech Processor}
\end{itemize}

\section{Ajustes Recomendados}

\subsection{Noise Reduction}
\begin{itemize}
    \item Activado (interruptor en posición ON)
    \item Función: Elimina ruido de fondo automáticamente
\end{itemize}

\subsection{Speech Processor}
\begin{itemize}
    \item \texttt{Gate Threshold}: Elimina ruido cuando no se habla
    \item \texttt{Compression}: Normaliza el volumen de la voz
    \item Los ajustes automáticos suelen ser efectivos
\end{itemize}

\section{Verificación}
\begin{itemize}
    \item Probar el micrófono hablando normalmente
    \item Verificar reducción de ruido ambiental
    \item Ajustar parámetros según necesidad
\end{itemize}

\section{Notas Técnicas}
\begin{itemize}
    \item EasyEffects funciona a nivel de sistema
    \param{Compatible con todas las aplicaciones}
    \item Requiere reinicio después de la instalación
    \item Consume recursos adicionales del sistema
\end{itemize}

\end{document}
